\documentclass[../main]{subfiles}
\begin{document}
\mode*

\chapter{实验}%
\label{cha:experiment}
\mode<representation>{%
  \section{实验}%
  \label{sec:experiment}
}

实验分为如下2个环节:

\mode<article>{%
  \section{单任务学习}%
  \label{sec:stl}
}

需要先在多任务学习开始前进行单任务学习,原因如下:

\begin{itemize}
  \item 为了客观的比较多任务学习与单任务学习相比到底有无提升,提升多少,需要
    同时得到单任务学习和多任务学习的结果;
  \item 为了更好的提升多任务模型的性能,需要将单任务模型的权重作为多任务模型
    训练的初始化权重。
\end{itemize}

\mode<article>{%
  \section{多任务学习}%
  \label{sec:mtl}
}

在完成了章节~\ref{sec:stl}的实验后,利用单任务学习得到的模型权重来初始化多任
务模型的权重。

\begin{frame}{实验}
  为了节省实验时间,对于耗时较高的语义分割、显著点检测只每隔10个时期和最后10个
  时期评估一次性能。

  \mode<article>{单、多任务性能随时间的变化如图~\ref{fig:result}。}单、多任务
  模型分别在 60、80 个时期后都出现了收敛的趋势。
\end{frame}

在1块 Nvidia GTX 1080 Ti 上实验的时间如表~\ref{tab:time}所示。原始实验数据可在
\url{https://github.com/Freed-Wu/graduation-design-report/releases/tag/data}找
到。

\begin{figure}[htbp]
  \centering
  \begin{subfigure}[htbp]{\linewidth}
    \begin{frame}[fragile]{语义分割}
      \centering
      \begin{tikzpicture}
        \begin{axis}[
            ylabel=IoU/\%,
            legend style={%
              at={(1, 0)},
              anchor=south west,
            },
            legend entries={%
              单任务在测试集,
              单任务在训练集,
              多任务在测试集,
              多任务在训练集,
              单任务前人最优结果,
              多任务前人最优结果,
          }]
          \addplot[blue, mark=triangle*] table {data/stl-semseg-test.dat};
          \addplot[blue, mark=triangle] table {data/stl-semseg-train.dat};
          \addplot[red, mark=triangle*] table {data/mti-semseg-test.dat};
          \addplot[red, mark=triangle] table {data/mti-semseg-train.dat};
          \addplot[blue, domain=0:60] {59.5};
          \addplot[red, domain=0:80] {64.3};
        \end{axis}
      \end{tikzpicture}
    \end{frame}
    \caption{语义分割}%
    \label{fig:semseg}
  \end{subfigure}

  \begin{subfigure}[htbp]{\linewidth}
    \centering
    \begin{frame}[fragile]{人体解析}
      \begin{tikzpicture}
        \begin{axis}[
            ylabel=IoU/\%,
            legend style={%
              at={(1, 0)},
              anchor=south west,
            },
            legend entries={%
              单任务在测试集,
              单任务在训练集,
              多任务在测试集,
              多任务在训练集,
              单任务前人最优结果,
              多任务前人最优结果,
          }]
          \addplot[blue, mark=square*] table {data/stl-human-test.dat};
          \addplot[blue, mark=square] table {data/stl-human-train.dat};
          \addplot[red, mark=square*] table {data/mti-human-test.dat};
          \addplot[red, mark=square] table {data/mti-human-train.dat};
          \addplot[blue, domain=0:60] {61.4};
          \addplot[red, domain=0:80] {62.1};
        \end{axis}
      \end{tikzpicture}
    \end{frame}
    \caption{人体解析}%
    \label{fig:human}
  \end{subfigure}

  \begin{subfigure}[htbp]{\linewidth}
    \centering
    \begin{frame}[fragile]{显著点检测}
      \begin{tikzpicture}
        \begin{axis}[
            ylabel=IoU/\%,
            legend style={%
              at={(1, 0)},
              anchor=south west,
            },
            legend entries={%
              单任务在测试集,
              单任务在训练集,
              多任务在测试集,
              多任务在训练集,
              单任务前人最优结果,
              多任务前人最优结果,
          }]
          \addplot[blue, mark=x] table {data/stl-sal-test.dat};
          \addplot[blue, mark=o] table {data/stl-sal-train.dat};
          \addplot[red, mark=x] table {data/mti-sal-test.dat};
          \addplot[red, mark=o] table {data/mti-sal-train.dat};
          \addplot[blue, domain=0:60] {67.3};
          \addplot[red, domain=0:80] {68.0};
        \end{axis}
      \end{tikzpicture}
    \end{frame}
    \caption{显著点检测}%
    \label{fig:sal}
  \end{subfigure}
  \caption{单、多任务性能随时期的变化}%
  \label{fig:result}
\end{figure}

\begin{table}[htbp]
  \centering
  \caption{实验时间}%
  \label{tab:time}
  \csvautobooktabular[respect percent]{tables/time.csv}
\end{table}

\begin{frame}{实验结果}
  本次实验结果与前人~\footnote{Vandenhende Simon}实验得到的最优结果在
  表~\ref{tab:result}中进行了对比。

  \begin{table}[htbp]
    \centering
    \caption{单任务学习与多任务学习在不同任务上结果的比较}%
    \label{tab:result}
    \csvautobooktabular[respect percent]{tables/result.csv}
  \end{table}
\end{frame}

\section{误差与改进}%
\label{sec:error}

\begin{frame}{误差分析}
  从表~\ref{tab:result}可以看出离最好结果相差2.6519、0.7433、0.598个百分点。分
  析原因可能是:

  \begin{itemize}
    \item 本实验只使用了3个任务进行多任务学习,而数据集支持其它任务,使用更多
      任务也许可以得到更好性能;
    \item 实验可能需要使用更多的技巧,例如集成、混合等;
    \item 一些改进(升级骨干网络、调整超参数等)可能对该数据集不合适;
    \item 实验结果本身就具有一定随机性,需要进行更多的重复实验。
  \end{itemize}
\end{frame}

\begin{frame}{改进策略}
  在模型的改进方案上,考虑了以下几种策略:

  \begin{itemize}
    \item 实验的骨干网络有意放弃了传统的残差网络,可以在此基础上进一步换用更
      好的网络;
    \item 本实验中出现了大量可供调节的超参数;
    \item 实验的多任务可以考虑经过合理的选择来避免负迁移现象的发生。
  \end{itemize}

  最终尝试后的结论如下:
  \begin{itemize}
    \item 考虑将骨干网从 v1 版本升级到了 v2 版本;
    \item 对于小批量梯度下降的批量大小、同时处理的任务个数而言,应当越小越好
      ,例如如果将批量大小从 32 改到 8 ,在相同的训练周期个数下可以增加约 1\%
      的性能提升;
    \item 如果增加更多任务(深度预测、边缘检测),语义分割性能下降的 2 个百分
      点依旧变化不大。
  \end{itemize}
\end{frame}

\mode<all>
\end{document}
