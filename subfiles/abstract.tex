\documentclass[../main]{subfiles}
\begin{document}
\mode*

\begin{abstract}
  从基于深度学习的方法在视觉理解任务的性能上超越了传统方法之后,俨然已经成为
  目前领域的主流方案。典型的深度学习方法通常是对每一个任务进行孤立的学习,然
  而这会导致过高的总参数量、内存与训练时间的消耗,而且还不一定能达到最终综合
  性能的最优。因此,对多个任务进行联合学习的方法已经成为研究的前沿热点,也是
  本文试图研究的方向。

  第一,本文针对传统的多任务学习的各种网络架构和方法进行了分析,总结和讨论了
  相关的优缺和结论。

  第二,在一的基础上本文仿照先前相关的研究设计了一个用于多任务学习的网络架构
  。

  第三,在二的基础上本文针对相关的数据集进行了详尽的实验与分析。

\begin{keyword}
  多任务学习 视觉理解
\end{keyword}
\end{abstract}

\begin{abstract*}
  After surpassing the traditional approach in visual understanding task
  performance from a deep learning-based approach, it has become a mainstream
  solution in the current field. A typical approach to deep learning is
  usually to learn in isolation for each task, but this can lead to excessive
  total parameter count, memory, and training time consumption, and may not
  necessarily result in optimal final comprehensive performance. Therefore,
  the method of joint learning on multiple tasks has become the forefront of
  research, and it is also the direction that this paper tries to study.

  First, this paper analyzes the various network architectures and methods of
  traditional multitasking learning, summarizes and discusses the relevant
  advantages and disadvantages and conclusions.

  Second, on the basis of one, this paper designs a network architecture for
  multitasking based on previous research

  Third, on the basis of the second, this paper carries on the detailed
  experiment and analysis for the relevant data set.

\begin{keyword*}
  Multi-Task Learning Visual Understanding
\end{keyword*}
\end{abstract*}

\mode<all>
\end{document}
