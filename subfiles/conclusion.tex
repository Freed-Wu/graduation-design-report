\documentclass[../main]{subfiles}
\begin{document}
\mode*

\chapter{总结}%
\label{cha:conclusion}

在本文中,我们首先介绍了多任务学习的概念、应用,简单回顾了多任务学习的历史;
然后叙述了多任务学习领域的两个热点问题:多任务网络架构的设计和多任务优化方法
的选择。接下来针对常见的两个数据集展开实验,分析探讨了多任务模型与单任务模型
相比,究竟是否有性能上的提升以及提升多少的问题。最终经过实验给出了可信的结论
。根据实验我们可以看出如下几点:

\begin{itemize}
  \item 多任务学习的性能与任务的信息(类型、标签)密切相关,这导致某一种模型
    可能适用于某一种情况(任务、数据)而在另一种情况下大打折扣,模型性能的提
    高也会与众多因素相关而难以独立分析;
  \item 对某些任务,多任务模型需要消耗的计算时间远超过其它任务,这限制了这些
    模型在实际场景的进一步应用。
\end{itemize}

尽管多任务学习已经被广泛使用在各个领域,但是最早的硬参数共享模型仍旧是神经网
络中多任务学习的主要基准。同时,我们对于该学习共享哪些信息、任务的相似性、任
务间的关系、任务的层次和多任务学习的收益等的了解仍旧十分有限,我们需要更多的
研究以理解这些问题。

\mode<all>
\end{document}
