\documentclass[../main]{subfiles}
\begin{document}
\mode*

\chapter{模型}%
\label{cha:model}

\section{数据集}%
\label{sec:dataset}

多任务模型在视觉理解领域常用的数据集如下:

\subsection{NYUD v2}%
\label{sub:nyud}

纽约大学深度 V2 数据集 NYU-Depth v2\cite{couprie2013indoor} 是由来自各种室内
场景的视频序列组成,这些场景由来自 Microsoft Kinect 的 RGB 和深度摄像机录制。
特点是:

\begin{itemize}
  \item 针对对齐的 RGB 和深度图像的 1449 个密集标记;
  \item 从3个城市拍摄的464个新场景;
  \item 407024 个新的无标签框架。
\end{itemize}

每个对象都标有类和实例编号。数据集包含如下组件:

\begin{itemize}
  \item Kinect 提供原始 RGB、深度和加速度计的数据。
  \item 视频数据的一部分子集附有密集的多类标签。这些数据经过预处理以填充缺失的深度标签。
  \item 用于操作数据和标签的有用工具。
\end{itemize}

\subsection{PASCAL}%
\label{sub:pascal}

\begin{frame}{PASCAL}
  PASCAL 上下文数据集~\cite{Everingham2010}是 PASCAL VOC 2010 检测挑战的延伸,
  它包含所有培训图像的像素标签。它包含超过 400 个类(包括原始的 20 个类加上
  PASCAL VOC 细分的背景),分为三个类别(对象、内容和混合)。此数据集的许多对
  象类别太稀少,因此,通常选择 59 个常用类的子集供使用。
\end{frame}

\section{建模}%
\label{sec:model}

\begin{frame}{任务}
  我们选择 PASCAL 作为实验的数据集。在该数据集支持的任务中,我们选取如下任务:

  \begin{description}
    \item[语义分割]通过查找属于某一目标的所有像素来识别图像中所有存在目标的内
      容以及位置;
    \item[人类部位分割]或者叫人体解析。作为语义分割任务的子任务,需要进一步对
      人类图像进行像素级的细粒度分割,例如,划分出身体部位和服装;
    \item[显著点检测]通过算法模拟人的视觉,提取图像的显著区域~\cite{730558}。
  \end{description}
\end{frame}

\begin{frame}{性能指标}
  我们选取的用来衡量模型性能的指标如下:

  \begin{definition}[IoU]
    预测结果$\hat{A}$与真实结果$A$的交、并集之比。
    式~\ref{eq:M}是式~\ref{eq:IoU}离散到像素单位上的计算方法,矩阵$\hat{M}, M$
    是$\hat{A}, A$的离散表示。这是一个正指标。
  \end{definition}

  \begin{align}
    \label{eq:IoU}
    \mathrm{IoU} & = \frac{A \cap \hat{A}}{A \cup \hat{A}} \\
    \label{eq:M}
                 & \approx \frac{M \odot \hat{M}}{M \oplus \hat{M}}
  \end{align}
\end{frame}

我们采用 HRNet 作为骨架。为了节省实验时间,使用了 HRNet 开发者提供的预训练模
型初始化模型参数。对于多任务模型,采用 MTI 的网络架构。为了方便与前人结果进行
对比,采用了与论文~\cite{9336293}相同的固定任务权重。

\mode<all>
\end{document}
