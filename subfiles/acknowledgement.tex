\documentclass[../main]{subfiles}
\begin{document}
\mode*

\begin{acknowledgement}
  \begin{frame}{致谢}
    \begin{center}
      \LARGE 谢谢!
    \end{center}
    下一页是参考文献。

    参考文献后的附录中是一些限于时间不便展开的细节。
  \end{frame}

  在写作本文时我要首先感谢我的指导老师。校内的指导老师帮助我完成了各种繁琐的
  相关手续工作,让我没有后顾之忧。校外的指导老师从定方向、选题、每周组会的讨
  论到最终的实验都给予了我很大的帮助。

  我也要感谢我的学长与学姐。在我遇到困难的时候是他们给了我经验和提示,告诉我
  查论文该去什么网站,什么样的期刊是好的,实验有什么需要注意的地方。正是在他
  们的帮助下我少走了很多弯路。

  我还要感谢在这一领域作出相关工作的研究者前辈们。艾萨克·牛顿说过:“我之所以
  能看的更远,是因为我站在巨人的肩膀上”。我不敢妄言比前辈们看得更远——毕竟我也
  不过是看着他们创造的杰作做了一点拙劣的模仿罢了——但如果没有他们的工作,让我
  从零开始创造这个研究领域的成果,哪怕是万分之一我也不能做到。越是了解他们的
  工作,我就越是深感自己在这一领域的渺小。数学知识的匮乏让我不得不不求甚解地
  接受我不能证明的结论,基本概念的欠缺让我不得不囫囵吞枣地理解我从未听过的定
  义,实验经验的不足让我不得不千辛万苦地解决我明知简单的问题。我衷心希望我可
  以在我选择的道路上一直走下去,戒除焦躁,脚踏实地,不急功近利,只求能解我内
  心好奇之苦、求知之欲。

  最后,从最初的一无所知到现在的一知半解,我也要感谢我自己的坚持。在最痛苦的
  时候,连梦中都要有人和我讨论为什么实验结果和前人差距如此之大,究竟是什么地
  方做的不对。我深知自己的弩钝,在有天赋的他人绝对不会犯错的地方做错误的事。
  但我仍想继续坚持下去。也许总有一天,我也会成为别人站立的肩膀,将会有人将他
  们看到的更远的风景分享与我。
\end{acknowledgement}

\mode<all>
\end{document}
